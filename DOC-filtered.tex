\documentclass{article}
\usepackage{xltxtra}
\usepackage[bloc,separateanswersheet,lang=FR]{automultiplechoice}
\usepackage{multicol}
\setmainfont{Linux Libertine O}
\geometry{paper=a4paper}
\usepackage{amssymb}\begin{document}
\def\AMCmakeTitle#1{\par\noindent\hrule\vspace{1ex}{\hspace*{\fill}\Large\bf #1\hspace*{\fill}}\vspace{1ex}\par\noindent\hrule\par\vspace{1ex}}
\AMCrandomseed{1527384}
\definecolor{amcboxcolor}{named}{black}
\AMCboxColor{amcboxcolor}
\makeatletter\AMC@outside@boxtrue\makeatother\element{grE}{
\insertPlainGroupunnamedA
}
\element{grE}{
\insertPlainGroupunnamedB
}
\element{grE}{
\insertPlainGroupunnamedC
}
\element{grE}{
\insertPlainGroupunnamedD
}
\def\insertPlainGroupgrE{\insertgroup{grE}}
\element{unnamedA}{
\begin{question}{Q001}
Le système de numération utilisé par les Egyptiens était
\begin{choices}
\wrongchoice{positionnel en base 10}
\correctchoice{non-{}positionnel en base 10}
\wrongchoice{positionnel en base 60}
\wrongchoice{non-{}positionnel en base 60}
\end{choices}
\end{question}
}
\element{unnamedA}{
\begin{question}{Q002}
Le système de numération utilisé par les Grecs était
\begin{choices}
\wrongchoice{positionnel en base 10}
\correctchoice{non-{}positionnel en base 10}
\wrongchoice{positionnel en base 60}
\wrongchoice{non-{}positionnel en base 60}
\end{choices}
\end{question}
}
\element{unnamedA}{
\begin{question}{Q003}
Quel est l’ordre chronologique correct (du plus ancien au plus moderne) ?
\begin{choices}
\wrongchoice{Pythagore, Archimède, Euclide}
\correctchoice{Pythagore, Euclide, Archimède}
\wrongchoice{Archimède, Pythagore, Euclide}
\wrongchoice{Archimède, Euclide, Pythagore}
\end{choices}
\end{question}
}
\element{unnamedA}{
\begin{question}{Q004}
Qui a introduit en Europe le système de numération arabe ?
\begin{choices}
\wrongchoice{Muhammad al-{}Khwarizmi}
\wrongchoice{Al Samaw’al}
\wrongchoice{Gérard de Crémone}
\correctchoice{Léonard de Pise}
\end{choices}
\end{question}
}
\element{unnamedA}{
\begin{question}{Q005}
Dans quel contexte les nombres complexes ont été utilisés pour la première fois ?
\begin{choices}
\wrongchoice{Dans la résolution d’équations du second degré}
\correctchoice{Dans la résolution d’équations du troisième degré}
\wrongchoice{Dans le problème géométrique de la trisectrice de l’angle}
\wrongchoice{Dans la représentation des rotations dans le plan}
\end{choices}
\end{question}
}
\element{unnamedA}{
\begin{question}{Q006}
Qui a introduit la représentation des nombres complexes dans le plan ?
\begin{choices}
\wrongchoice{Leonhard Euler}
\correctchoice{Car Friedrich Gauss}
\wrongchoice{René Descartes}
\wrongchoice{Georg Cantor}
\end{choices}
\end{question}
}
\element{unnamedA}{
\begin{question}{Q007}
Laquelle de ces suites correspond aux premiers nombres triangulaires ?
\begin{choices}
\wrongchoice{0, 1, 2, 3, 5, 8, 13}
\wrongchoice{1, 2, 4, 9, 16, 25}
\correctchoice{1, 3, 6, 10, 15, 21}
\wrongchoice{0, 2, 5, 9, 14, 20}
\end{choices}
\end{question}
}
\element{unnamedA}{
\begin{question}{Q008}
Quel philosophe grec défendait l’idée que le mouvement est une illusion ?
\begin{choices}
\correctchoice{Parménide d’Elée}
\wrongchoice{Anaximandre de Milet}
\wrongchoice{Démocrite d’Abdère}
\wrongchoice{Archimède de Syracuse}
\end{choices}
\end{question}
}
\element{unnamedA}{
\begin{question}{Q009}
Un infini ne pouvant pas être mis en bijection avec les nombres naturels est appelé
\begin{choices}
\wrongchoice{un infini actuel}
\wrongchoice{un infini potentiel}
\wrongchoice{un infini dénombrable}
\correctchoice{un infini indénombrable}
\end{choices}
\end{question}
}
\element{unnamedA}{
\begin{question}{Q010}
Le paradoxe de Galilée montre
\begin{choices}
\wrongchoice{l’existence de processus qui s’achèvent mais qui n’ont pas d’étape finale}
\wrongchoice{l’existence de processus qui commencent mais qui n’ont pas d’étape initiale}
\correctchoice{l’existence d’objets dont une partie est aussi grande que le tout}
\wrongchoice{l’existence d’une somme infinie de termes donnant un résultat fini}
\end{choices}
\end{question}
}
\element{unnamedA}{
\begin{question}{Q011}
Quel est l’ordre chronologique correct (du plus ancien au plus moderne) ?
\begin{choices}
\correctchoice{Bolzano, Cantor, Godel}
\wrongchoice{Bolzano, Godel, Cantor}
\wrongchoice{Cantor, Bolzano, Godel}
\wrongchoice{Godel, Cantor, Bolzano}
\end{choices}
\end{question}
}
\element{unnamedA}{
\begin{question}{Q012}
Soient E et F deux ensembles tels qu’il existe une surjection de E vers F. Alors
\begin{choices}
\wrongchoice{La cardinalité de E est strictement plus grande que la cardinalité de F}
\correctchoice{La cardinalité de E est plus grande ou égale à la cardinalité de F}
\wrongchoice{La cardinalité de E est strictement plus petite que la cardinalité de F}
\wrongchoice{La cardinalité de E est plus petite ou égale à la cardinalité de F}
\end{choices}
\end{question}
}
\element{unnamedA}{
\begin{question}{Q013}
Soient E et F deux ensembles tels qu’il existe une injection de E vers F. Alors
\begin{choices}
\wrongchoice{La cardinalité de E est strictement plus grande que la cardinalité de F}
\wrongchoice{La cardinalité de E est plus grande ou égale à la cardinalité de F}
\wrongchoice{La cardinalité de E est strictement plus petite que la cardinalité de F}
\correctchoice{La cardinalité de E est plus petite ou égale à la cardinalité de F}
\end{choices}
\end{question}
}
\element{unnamedA}{
\begin{question}{Q014}
L’hypothèse du continu dit
\begin{choices}
\wrongchoice{qu’il y a une bijection entre les nombres réels et les nombres complexes}
\wrongchoice{qu’il n’y a pas d’infini plus grand que celui des nombres réels}
\correctchoice{que l’infini indénombrable le plus petit est celui des nombres réels}
\wrongchoice{que l’ensemble des parties de $\mathbb{N}$ est $\mathbb{R}$}
\end{choices}
\end{question}
}
\def\insertPlainGroupunnamedA{\noindent \AMCmakeTitle{Histoire des Mathématiques}\vspace{1.5ex}\par
\insertgroup[10]{unnamedA}\vspace{1.5ex}\par
}
\element{unnamedB}{
\begin{question}{Q015}
Quel est l’ordre chronologique correct (du plus ancien au plus moderne) ?
\begin{choices}
\correctchoice{Aristote, Hipparque, Ptolémée}
\wrongchoice{Aristote, Ptolémée, Hipparque}
\wrongchoice{Hipparque, Aristote, Ptolémée}
\wrongchoice{Hipparque, Ptolémée, Aristote}
\end{choices}
\end{question}
}
\element{unnamedB}{
\begin{question}{Q016}
Comment s’appelle l’œuvre majeure de Ptolémée ?
\begin{choices}
\correctchoice{L’Almageste}
\wrongchoice{Le Banquet des Cendres}
\wrongchoice{Le Ciel}
\wrongchoice{Le Timée}
\end{choices}
\end{question}
}
\element{unnamedB}{
\begin{questionmult}{Q017}\scoring{haut=2}
Parmi les affirmations suivantes, avec lesquelles serait d’accord Hipparque ? (Plusieurs bonnes réponses possibles)
\begin{choices}
\correctchoice{Le mouvement du Soleil est circulaire et uniforme}
\wrongchoice{Le centre de l’orbite du Soleil est la Terre}
\wrongchoice{Le mouvement des planètes est circulaire et uniforme}
\correctchoice{Le centre de l’orbite des étoiles est la Terre}
\end{choices}
\end{questionmult}
}
\element{unnamedB}{
\begin{question}{Q018}
Une théorie avancée par les astronomes hellénistiques a motivé Copernic à introduire le système héliocentrique. Laquelle ?
\begin{choices}
\wrongchoice{La théorie des épicycles}
\wrongchoice{La théorie des excentricités}
\correctchoice{La théorie du point équant}
\wrongchoice{La théorie du déférent}
\end{choices}
\end{question}
}
\element{unnamedB}{
\begin{question}{Q019}
Quel est l’ordre chronologique correct (du plus ancien au plus moderne) ?
\begin{choices}
\correctchoice{Bruno, Galilée, Leibniz}
\wrongchoice{Bruno, Leibniz, Galilée}
\wrongchoice{Galilée, Bruno, Leibniz}
\wrongchoice{Leibniz, Galilée, Bruno}
\end{choices}
\end{question}
}
\element{unnamedB}{
\begin{question}{Q020}
Quelle expérience a mis en évidence la relativité du mouvement uniforme ?
\begin{choices}
\wrongchoice{L’expérience du seau d’eau de Newton}
\wrongchoice{L’expérience de Mickelson-{}Morlay}
\correctchoice{L’expérience du bateau de Gassendi}
\wrongchoice{L’expérience de la pomme de Galilée}
\end{choices}
\end{question}
}
\element{unnamedB}{
\begin{question}{Q021}
Les équations de l’électromagnétisme de Maxwell ont été écrites
\begin{choices}
\wrongchoice{entre 1750 et 1800}
\wrongchoice{entre 1800 et 1850}
\correctchoice{entre 1850 et 1900}
\wrongchoice{entre 1900 et 1950}
\end{choices}
\end{question}
}
\def\insertPlainGroupunnamedB{\noindent \AMCmakeTitle{Histoire de l'Espace et du Temps}\vspace{1.5ex}\par
\insertgroup[5]{unnamedB}\vspace{1.5ex}\par
}
\element{unnamedC}{
\begin{questionmult}{Q022}\scoring{haut=2}
Suite aux deux amphithéâtres consacrés à l’introduction générale de l’histoire des sciences, quelle définition faut-{}il retenir de cette discipline ?
\begin{choices}
\wrongchoice{C’est le récit chronologique des découvertes scientifiques}
\correctchoice{C’est une enquête rigoureuse sur la progression des modes de connaissance}
\wrongchoice{C’est le récit  des grands changements scientifiques dans l’histoire}
\correctchoice{C’est l’étude de l’évolution des paradigmes scientifiques, de la pensée}
\end{choices}
\end{questionmult}
}
\element{unnamedC}{
\begin{questionmult}{Q023}\scoring{haut=2}
A partir de quel moment considère-{}ton qu’il y a réellement science ?
\begin{choices}
\wrongchoice{Dès lors que nous sommes en capacité de raisonner}
\correctchoice{Dès lors qu’un contenu de connaissance fait l’objet d’une formalisation systématique}
\wrongchoice{Dès lors qu’un savoir fait l’objet d’une croyance partagée}
\correctchoice{Dès lors que des connaissances scientifiques, vérifiées formellement et/ou expérimentalement, sont universellement tenues pour vraies}
\end{choices}
\end{questionmult}
}
\element{unnamedC}{
\begin{questionmult}{Q024}\scoring{haut=2}
Platon, dans sa République, distingue deux mondes : comment les nomme-{}t-{}ils ?
\begin{choices}
\wrongchoice{Le monde de la lumière}
\correctchoice{Le monde sensible}
\wrongchoice{Le monde de la matière}
\correctchoice{Le monde intelligible}
\end{choices}
\end{questionmult}
}
\element{unnamedC}{
\begin{question}{Q025}
Platon se sert d’une allégorie pour illustrer sa distinction de deux mondes : comment se nomme-{}t-{}elle ?
\begin{choices}
\wrongchoice{De la grotte}
\wrongchoice{Des esclaves}
\correctchoice{De la caverne}
\wrongchoice{De la philosophie}
\end{choices}
\end{question}
}
\element{unnamedC}{
\begin{questionmult}{Q026}\scoring{haut=2}
Pour Platon, qu’est-{}ce qui nous permet d’atteindre la connaissance vraie ?
\begin{choices}
\correctchoice{Les Idées}
\correctchoice{Les mathématiques}
\wrongchoice{Les sens}
\wrongchoice{L’imagination}
\end{choices}
\end{questionmult}
}
\element{unnamedC}{
\begin{question}{Q027}
Quelle formule était inscrite sur le fronton de l’Académie de Platon à Athènes ?
\begin{choices}
\wrongchoice{Que nul n’entre ici s’il n’est philosophe}
\wrongchoice{Que nul n’entre ici s’il n’est mathématicien}
\correctchoice{Que nul n’entre ici s’il n’est géomètre}
\wrongchoice{Que nul n’entre ici s’il n’est sophiste}
\end{choices}
\end{question}
}
\element{unnamedC}{
\begin{questionmult}{Q028}\scoring{haut=2}
Pour quelle raison le paradigme platonicien des deux mondes a-{}t-{}il freiné l’essor des sciences ?
\begin{choices}
\wrongchoice{Parce qu’il promeut la nécessité, pour la science, d’être en étroite relation avec la matière, la nature, le monde concret}
\correctchoice{Parce que seul l’intelligible pur, soit l’abstraction seule, peut accéder au vrai}
\wrongchoice{Parce que l’expérience sensible était, pour lui, au cœur de la recherche de la vérité}
\correctchoice{Parce qu’il rompt tout lien possible entre les réalités intelligibles et les réalités sensibles}
\end{choices}
\end{questionmult}
}
\element{unnamedC}{
\begin{questionmult}{Q029}\scoring{haut=2}
Dans le livre VII de La République, Platon met en place un système de valeurs qui a structuré et structure encore notre image du monde. Relevez deux couples platoniciens de valeurs synonymes parmi les quatre proposés ici.
\begin{choices}
\correctchoice{Intelligence/Beauté}
\wrongchoice{Intelligence/Cruauté}
\correctchoice{Sensibilité/Ignorance}
\wrongchoice{Sensibilité/Vérité}
\end{choices}
\end{questionmult}
}
\element{unnamedC}{
\begin{question}{Q030}
En quoi la position scientifique d’Aristote se distingue-{}telle de la position de Platon quant à la connaissance ?
\begin{choices}
\wrongchoice{Aristote réhabilite la rhétorique}
\wrongchoice{Aristote crée la logique formelle}
\correctchoice{Avec Aristote, ce qui est observé prime sur la théorie}
\wrongchoice{Avec Aristote, la perception sensible n’a plus droit de cité}
\end{choices}
\end{question}
}
\element{unnamedC}{
\begin{questionmult}{Q031}\scoring{haut=2}
Les grandes ruptures épistémologiques sont influencées par des causes extérieures aux sciences. Quelles sont ces causes ?
\begin{choices}
\correctchoice{Un environnement socioculturel, politique et économique}
\correctchoice{Les avancées techniques}
\wrongchoice{Le hasard}
\wrongchoice{La religion}
\end{choices}
\end{questionmult}
}
\element{unnamedC}{
\begin{questionmult}{Q032}\scoring{haut=2}
Que faut-{}il entendre par « révolution copernicienne » ?
\begin{choices}
\wrongchoice{Que le système géocentrique est réhabilité}
\correctchoice{Que le système héliocentrique supplante le système géocentrique}
\correctchoice{Que l’inversion de la relation sujet/objet produit un nouveau paradigme philosophique}
\wrongchoice{Que Copernic fait sa révolution}
\end{choices}
\end{questionmult}
}
\element{unnamedC}{
\begin{question}{Q033}
Que faut-{}il entendre par « sciences molles » ?
\begin{choices}
\wrongchoice{Des sciences non structurées}
\wrongchoice{Des sciences non déterminées}
\correctchoice{Des sciences dont l’objet est instable, changeant}
\wrongchoice{Des sciences flasques}
\end{choices}
\end{question}
}
\element{unnamedC}{
\begin{questionmult}{Q034}\scoring{haut=2}
Parmi ces quatre propositions, relevez deux blessures narcissiques faites à l’homme.
\begin{choices}
\correctchoice{La conscience, contrairement à ce que pensait Descartes, n’est plus reine en sa demeure (théorie de l’inconscient)}
\correctchoice{L’homme ne jouit plus d’une place particulière dans l’ordre de la création (théorie de la sélection naturelle, Darwin)}
\wrongchoice{Dieu est le créateur, l’homme sa créature ; Dieu est tout puissant, l’homme est faible par nature}
\wrongchoice{« La terre ne forme qu’une parcelle insignifiante de système cosmique » (Freud, à propos du démenti de Copernic)}
\end{choices}
\end{questionmult}
}
\element{unnamedC}{
\begin{question}{Q035}
Les Grecs avaient l’intuition de l’inconscient, porté notamment par les Chœurs dans les tragédies, mais qui l’a systématisé et formalisé, élevé au rang de théorie scientifique ?
\begin{choices}
\wrongchoice{Leibniz}
\wrongchoice{Lacan}
\wrongchoice{Darwin}
\correctchoice{Freud}
\end{choices}
\end{question}
}
\def\insertPlainGroupunnamedC{\noindent \AMCmakeTitle{Introduction générale à l'Histoire des Sciences}\vspace{1.5ex}\par
\insertgroup[10]{unnamedC}\vspace{1.5ex}\par
}
\element{unnamedD}{
\begin{question}{Q036}
Quelle est la principale innovation apportée par le Métier à tisser Jacquard ?
\begin{choices}
\correctchoice{c'est la première machine qui utilise un modèle codé par carte perforée}
\wrongchoice{c'est une des premières machines à calculer industrialisées}
\wrongchoice{c'est la première machine programmable}
\wrongchoice{c'est la première machine qui produit des cartes perforées}
\end{choices}
\end{question}
}
\element{unnamedD}{
\begin{question}{Q037}
Quelle source d'énergie est utilisée par les premières horloges ?
\begin{choices}
\correctchoice{l'énergie gravitationnelle}
\wrongchoice{l'énergie thermique}
\wrongchoice{l'énergie chimique}
\wrongchoice{l'énergie éolienne}
\end{choices}
\end{question}
}
\element{unnamedD}{
\begin{question}{Q038}
Quel dispositif est utilisé par le sémaphore de Chappe pour transmettre des informations ?
\begin{choices}
\correctchoice{des bras articulés}
\wrongchoice{un code binaire sur 5 bits}
\wrongchoice{un encodage avec des traits et des points}
\wrongchoice{des signaux lumineux}
\end{choices}
\end{question}
}
\element{unnamedD}{
\begin{questionmult}{Q039}\scoring{haut=2}
Quel(s) scientifiques ont élaboré la théorie des algorithmes ? (plusieurs réponses possibles)
\begin{choices}
\correctchoice{Alan Turing}
\correctchoice{Alonzo Church}
\wrongchoice{John von Neumann}
\wrongchoice{Ada Lovelace}
\end{choices}
\end{questionmult}
}
\element{unnamedD}{
\begin{question}{Q040}
Quel scientifique a défini l'architecture des machines sur le modèle CPU -{} Mémoire -{} UAL ?
\begin{choices}
\correctchoice{John von Neumann}
\wrongchoice{Charles Babbage}
\wrongchoice{Konrad Zuse}
\wrongchoice{John Eckert}
\end{choices}
\end{question}
}
\element{unnamedD}{
\begin{question}{Q041}
L'ordinateur ENIAC, mis au point en 1945, est une machine :
\begin{choices}
\correctchoice{électronique utilisant des tubes à vide}
\wrongchoice{électronique utilisant des transistors}
\wrongchoice{mécanique utilisant des roues crantées}
\wrongchoice{électromécanique réalisant les quatre opérations}
\end{choices}
\end{question}
}
\element{unnamedD}{
\begin{question}{Q042}
Ada Lovelace a indiqué, à propos des machines programmables, que :
\begin{choices}
\correctchoice{elles pouvaient faire bien plus que du simple calcul numérique}
\wrongchoice{elles pouvaient uniquement faire du calcul numérique}
\wrongchoice{elles pouvaient produire des résultats probabilistes}
\wrongchoice{elles pouvaient communiquer entre elles avec des données binaires}
\end{choices}
\end{question}
}
\def\insertPlainGroupunnamedD{\noindent \AMCmakeTitle{Histoire de l'Informatique}\vspace{1.5ex}\par
\insertgroup[5]{unnamedD}}
\onecopy{5}{
\begin{center}\bf\large \AMCmakeTitle{\textbf{L2 2018/2019 -{} FHS301 -{} HISTOIRE DES SCIENCES}}

\textbf{le 18/12/2018}

\textbf{Documents et calculatrices non autorisés}
\vspace{1\baselineskip}\end{center}

\vspace{3\baselineskip}
\AMCmakeTitle{SUJET DU QUESTIONNAIRE}

Ne répondez pas directement sur ce sujet. Utilisez la \textbf{FICHE DE REPONSE} pour y reporter vos réponses.

Les questions comportant le symbole $\clubsuit$ sont des questions pour lesquelles il y a plusieurs bonnes réponses possibles.
\vspace{1\baselineskip}

\vspace{4mm}\noindent\hrule


\shufflegroup{grE}
\shufflegroup{unnamedA}
\shufflegroup{unnamedB}
\shufflegroup{unnamedC}
\shufflegroup{unnamedD}
\vspace{2ex}

\insertPlainGroupgrE

\AMCcleardoublepage

\AMCformBegin
\begin{center}\bf\large FICHE DE REPONSE

Les réponses doivent être fournies uniquement sur cette feuille

Les cases doivent être \textbf{remplies} pour être prises en compte\end{center}

{\setlength{\parindent}{0pt}\hspace*{\fill}\AMCcode{student.number}{8}\hspace*{\fill}\begin{minipage}[b]{5.8cm}$\longleftarrow{}$\hspace{0pt plus 1cm}Veuillez coder votre numéro étudiant Efrei dans la grille de gauche, et indiquer vos nom, prénom et numéro étudiant ci-dessous\vspace{3ex}

\hfill\namefield{\fbox{\begin{minipage}{.9\linewidth}Nom - Prénom  -  n° étudiant

\vspace*{.5em}\dotfill

\vspace*{.5em}\dotfill

\vspace*{.5em}\dotfill
\vspace*{1mm}
\end{minipage}
}}\hfill\vspace{5ex}\end{minipage}\hspace*{\fill}

}\vspace{4mm}
\vspace{4mm}\noindent\hrule


\begin{multicols}{2}
\AMCform
\end{multicols}
\AMCcleardoublepage
}
\end{document}
